\documentclass[a4paper]{article}

\usepackage[utf8]{inputenc}
\usepackage[margin=1.5cm]{geometry}
\usepackage{syntax}
\usepackage{rail}

\setlength{\grammarindent}{4em}

\begin{document}

\title{Syntax Explanation and Definition of SEN}
\author{Zirak}
\date{\today}

\maketitle

\tableofcontents

\section{The Complete Syntax}

\begin{grammar}
	<program> $\to$ <value>
		\alt $\epsilon$
		
	<value> $\to$ <atom>
		\alt <symbol>
		\alt <list>
		\alt <plist>
		
	<atom> $\to$ `nil'
		\alt `t'
		\alt <anything>
		
	<symbol> $\to$ `:' <atom>
	
	<list> $\to$ `(' <list-values> `)'

	<list-values> $\to$ <value> <list-values>
		\alt $\epsilon$
		
	<plist> $\to$ `(' <pairs> `)'

	<pairs> $\to$ <symbol> <value> <pairs>
		\alt $\epsilon$
\end{grammar}

\section{Individual Components}

\subsection{Value}
\begin{rail}
	value : (atom | symbol | list | plist)
\end{rail}

\begin{grammar}
	<value> $\to$ <atom>
		\alt <symbol>
		\alt <list>
		\alt <plist>
\end{grammar}

A $value$ is any of the possible SEN structures.

\subsection{Atom}
\begin{rail}
	atom : 'nil' | 't' | anything
\end{rail}

\begin{grammar}
	<atom> $\to$ `nil'
		\alt `t'
		\alt <anything>
\end{grammar}

An $atom$ is any of the special constructs $nil$ or $t$, or any combination of characters, excluding the space character and parentheses $()$. In addition, an $atom$ may not begin with the colon, $:$.

\subsection{Symbol}
\begin{rail}
	symbol : ':' atom
\end{rail}

\begin{grammar}
	<symbol> $\to$ `:' <atom>
\end{grammar}

A $symbol$ is a literal value. While an $atom$ may be subject to interpretations ($t$ may turn to $true$ in a target language), a $symbol$ will always appear as-is.


\subsection{List}
\begin{rail}
	list : '(' ( () + value ) ')'
\end{rail}

\begin{grammar}
	<list> $\to$ `(' <list-values> `)'

	<list-values> $\to$ <value> <list-values>
		\alt $\epsilon$
\end{grammar}

A $list$ is one or more $value$s, separated by spaces. They do not have to be homogeneous; that is, you can mix up the value types. You may arbitrarily nest lists to easily create complex structures.


\subsection{Property-List}
\begin{rail}
	plist : '(' (() + (symbol value)) ')'
\end{rail}

\begin{grammar}
	<plist> $\to$ `(' <pairs> `)'

	<pairs> $\to$ <symbol> <value> <pairs>
		\alt $\epsilon$
\end{grammar}

$p-lists$, or $property-lists$, can be considered a poor man's hash-table. They are made of one or more $key => value$ pairs, where the key must be a $symbol$, and the value may be any $value$ allowed in the language. The $key$ and $value$ are separated by a space, and so are each pair.

\end{document}