\documentclass[a4paper]{article}

\usepackage[utf8]{inputenc}
\usepackage[margin=1.5cm]{geometry}
\usepackage{syntax}
\usepackage{rail}

\setlength{\grammarindent}{4em}

\begin{document}

\title{Syntax Definition and Explanation of SEN}
\author{Zirak}
\date{\today}

\maketitle

\tableofcontents

\section{The Complete Syntax}

\begin{grammar}
	<program> $\to$ <value>
		\alt $\epsilon$

	<value> $\to$ <atom>
		\alt <symbol>
		\alt <string>
		\alt <list>
		\alt <plist>

	<atom> $\to$ `nil'
		\alt `t'
		\alt <anything>

	<symbol> $\to$ `:' <atom>

	<string> $\to$ `"' <chars> `"'

	<chars> $\to$ <unicode-char> <chars>
		\alt <escaped-char> <chars>
		\alt $\epsilon$

	<escaped-char> $\to$ `\\b'
		\alt `\\f'
		\alt `\\n'
		\alt `\\r'
		\alt `\\t'
		\alt `\\u' <hex-digit> <hex-digit> <hex-digit> <hex-digit>

	<list> $\to$ `(' <list-values> `)'

	<list-values> $\to$ <value> <list-values>
		\alt $\epsilon$

	<plist> $\to$ `(' <pairs> `)'

	<pairs> $\to$ <symbol> <value> <pairs>
		\alt $\epsilon$

	<comment> $\to$ `;' <comment-chars>

	<comment-chars> $\to$ <not-line-terminator> <comment-chars>
		\alt $\epsilon$
\end{grammar}

\section{Individual Components}

\subsection{Values}
\begin{rail}
	value : (atom | symbol | string | list | plist)
\end{rail}

\begin{grammar}
	<value> $\to$ <atom>
		\alt <symbol>
		\alt <string>
		\alt <list>
		\alt <plist>
\end{grammar}

A $value$ is any of the possible SEN structures.

\subsection{Atoms}
\begin{rail}
	atom : 'nil' | 't' | anything
\end{rail}

\begin{grammar}
	<atom> $\to$ `nil'
		\alt `t'
		\alt <anything>
\end{grammar}

An $atom$ is any of the special constructs $nil$ or $t$, or any combination of characters, excluding the space character and parentheses $()$. In addition, an $atom$ may not begin with the colon, $:$.

The $nil$ value is akin to $null$ or $none$ is many other programming languages. It is also used as the de-facto $false$. $t$ is akin to $true$.

\subsection{Symbols}
\begin{rail}
	symbol : ':' atom
\end{rail}

\begin{grammar}
	<symbol> $\to$ `:' <atom>
\end{grammar}

A $symbol$ is a literal value. While an $atom$ may be subject to interpretations (for example, $t$ may turn to $true$ in a target language), a $symbol$ will always appear as-is.

\subsection{Strings}
\begin{rail}
	string : '"' (() + (
		unicodeChar | (
		'\symbol{92}' (
			[backspace] 'b' |
			[form feed] 'f' |
			[line feed] 'n' |
			[carriage return] 'r' |
			[tab] 't' |
			('u' fourHexDigits))))
	) '"'
\end{rail}

\begin{grammar}
	<string> $\to$ `"' <char> `"'

	<char> $\to$ <unicode-char> <char>
		\alt <escaped-char> <char>
		\alt $\epsilon$

	<escaped-char> $\to$ `\\b'
		\alt `\\f'
		\alt `\\n'
		\alt `\\r'
		\alt `\\t'
		\alt `\\u' <hex-digit> <hex-digit> <hex-digit> <hex-digit>
\end{grammar}

A $string$ is what you may be familiar with from the C family. $Strings$ are delimited by the double-quote character. Any Unicode character may be written inside the double-quotes, with the exception of the double-quote and the backslash, which must be escaped. To escape a character, one writes the backslash character, followed by the desired character. For example: $\backslash "$, which results in the literal double-quote character; $\backslash p$, which results in the character $p$.

Several common characters which may be difficult to write directly are represented by an escape-sequence:

\begin{center}
\begin{tabular}{l c r}
	Sequence	   & Meaning		 & Unicode \\
	\hline
	$\backslash b$ & backspace		 & \texttt{U+0008} \\
	$\backslash f$ & form-feed		 & \texttt{U+000C} \\
	$\backslash n$ & line-feed		 & \texttt{U+000A} \\
	$\backslash r$ & carriage-return & \texttt{U+000D} \\
	$\backslash t$ & tab			 & \texttt{U+0009}
\end{tabular}
\end{center}

Another escape-sequence, $\backslash u$, may be used to represent unicode values inside the range \texttt{U+0000} $\to$ \texttt{U+FFFF}, as such $\backslash u XXXX$ where each $X$ is a hexadecimal digit (case insensitive). For instance, the tab character \texttt{U+0009} may be written as $\backslash u 0009$.

\subsection{Lists}
\begin{rail}
	list : '(' ( () + value ) ')'
\end{rail}

\begin{grammar}
	<list> $\to$ `(' <list-values> `)'

	<list-values> $\to$ <value> <list-values>
		\alt $\epsilon$
\end{grammar}

A $list$ is one or more $value$s, separated by spaces. They do not have to be homogeneous; that is, you can mix up the value types. You may arbitrarily nest lists to easily create complex structures.


\subsection{Property-Lists}
\begin{rail}
	plist : '(' (() + (symbol value)) ')'
\end{rail}

\begin{grammar}
	<plist> $\to$ `(' <pairs> `)'

	<pairs> $\to$ <symbol> <value> <pairs>
		\alt $\epsilon$
\end{grammar}

$p-lists$, or $property-lists$, can be considered a poor man's hash-table. They are made of one or more $key => value$ pairs, where the key must be a $symbol$, and the value may be any $value$ allowed in the language. The $key$ and $value$ are separated by a space, and so are each pair. Like regular $lists$, $p-lists$ are heterogenous.

\subsection{Comments}
\begin{rail}
	comment : ';' (()+ not[line-terminator])
\end{rail}

\begin{grammar}
	<comment> $\to$ `;' <comment-chars>

	<comment-chars> $\to$ <not-line-terminator> <comment-chars>
		\alt $\epsilon$
\end{grammar}

A $comment$ may be inserted at any point in the program, except inside a string. Its contents are ignored by the parser. The comment spans from the beginning of the semi-colon $;$ until a line-terminator is met (EOL or EOF).

\end{document}
